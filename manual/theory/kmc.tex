\section{Master equation}
\label{sec:kmc}

Having determined the list of conjugated segments (hopping sites) and charge transfer rates between them, the next task is to solve the master equation which describes the time evolution of the system
%
\begin{equation}
\label{equ:master}
\frac{\partial P_\alpha}{\partial t} = \sum_{\beta} P_\beta \Omega_{\beta \alpha} - 
\sum_{\beta} P_\alpha \Omega_{\alpha \beta},
\end{equation}
%
where $P_\alpha$ is the probability of the system to be in a state $\alpha$ at time $t$ and $\Omega_{\alpha \beta}$ is the transition rate from state $\alpha$ to state $\beta$. A state $\alpha$ is specified by a set of site occupations, $\left\{ \alpha_i \right\}$, where $\alpha_i = 1 (0)$ for an occupied (unoccupied) site $i$, and the matrix $\hat{\Omega}$ can be constructed from rates $\omega_{ij}$.

The solution of \equ{master} is be obtained by using kinetic Monte Carlo (KMC) methods. KMC explicitly simulates the dynamics of charge carriers by constructing a Markov chain in state space and can find both stationary and transient solutions of the master equation. The main advantage of KMC is that only states with a direct link to the current state need to be considered at each step. Since these can be constructed solely from current site occupations, extensions to multiple charge carriers (without the mean-field approximation), site-occupation dependent rates (needed for the explicit treatment of Coulomb interactions), and different types of interacting particles and processes, are straightforward. To optimize memory usage and efficiency, a combination of the variable step size method~\cite{bortz_new_1975} and the first reaction method is implemented.