\subsection{Spatial correlations of energetic disorder}
\label{sec:eanalyze}

\index{site energy!spatial correlation}
Long-range, e.g. electrostatic and polarization, interactions often result in spatially correlated disorder~\cite{dunlap_charge-dipole_1996}, which affects the onset of the mobility-field (Poole-Frenkel) dependence~\cite{derrida_velocity_1983,novikov_essential_1998,nagata_atomistic_2008}.    

To quantify the degree of correlation, one can calculate the spatial correlation function of $E_i$ and $E_j$ at a distance $r_{ij}$
\begin{equation}
\label{equ:cf}
C(r_{ij}) = \frac{  \langle \left( E_i-\langle E\rangle \right)
                   \left( E_j-\langle E\rangle \right)\rangle}
                   {\langle\left( E_i -\langle E\rangle \right)^2\rangle},
\end{equation}
where $\langle E\rangle$ is the average site energy. $C(r_{ij})$ is zero if $E_i$ and $E_j$ are uncorrelated and $1$ if they are fully correlated. For a system of randomly oriented point dipoles, the correlation function decays as $1/r$ at large distances~\cite{novikov_cluster_1995}.

For systems with spatial correlations, variations in site energy differences, $\Delta E_{ij}$, of pairs of molecules from the neighbor list are smaller than variations in site energies, $E_i$, of all individual molecules. Since only neighbor list pairs affect transport, the distribution of $\Delta E_{ij}$ rather than that of individual site energies, $E_i$, should be used to characterize energetic disorder.

Note that the \calc{eanalyze} calculator takes into account {\em all} contributions to the site energies 
\votcacommand{Analyze distribution and correlations of site energeies}{\cmdeana}

