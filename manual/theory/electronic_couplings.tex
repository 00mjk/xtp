\section{Electronic coupling elements}
\label{sec:transfer_integrals}

The electronic transfer integral $J_{ij}$ entering the Marcus rates in \equ{marcus} is defined as
\begin{equation}
   J_{ij} = \left\langle \phi^i \left\vert \hat{H} \right\vert \phi^j \right\rangle ,
\label{equ:TI}
\end{equation}
where $\phi^i$ and $\phi^j$ are diabatic wavefunctions, localized on molecule $i$ and $j$ respectively, participating in the charge transfer, and $\hat{H}$ is the Hamiltonian of the formed dimer. Within the frozen-core approximation, the usual choice for the diabatic wavefunctions $\phi^i$  is the highest occupied molecular orbital (HOMO) in case of hole transport, and the lowest unoccupied molecular orbital (LUMO) in the case of electron transfer, while $\hat{H}$ is an effective single particle Hamiltonian, e.g. Fock or Kohn-Sham operator of the dimer. As such, $J_{ij}$ is a measure of the strength of the electronic coupling of the frontier orbitals of monomers mediated by the dimer interactions. Intrinsically, the transfer integral is very sensitive to the molecular arrangement, i.e. the distance and the mutual orientation of the molecules participating in charge transport. Since this arrangement can also be significantly influenced by
static and/or dynamic disorder~\cite{hutchison_hopping_2005,kirkpatrick_columnar_2008,troisi_charge_2009,vehoff_charge_2010-1,vehoff_charge_2010-2},
it is essential to calculate $J_{ij}$ explicitly for each hopping pair within a realistic morphology. Considering that the number of dimers for which \equ{TI} has to be evaluated is proportional to the number of molecules times their coordination number, computationally efficient and at the same time quantitatively reliable schemes are required.

In general, information about three objects is needed: the two monomer wave functions $\phi^i$ and $\phi^j$, and the dimer interaction Hamiltonian $\hat{H}$.  

\newcommand{\integrals}{\hyperref[calc:integrals]{\texttt{integrals}}\xspace}

\subsection{Molecular Orbital Overlap}

\moo can be used both in a sandalone mode and as a \calculator of the \votcact. \moo constructs the Fock operator of a dimer from the  molecular orbitals of monomers by translating and rotating the orbitals and therefore requires the optimized geometry of the molecule and the projection coefficients of the molecular on atomic orbitals. 


\subsubsection{Standalone mode}
For a standalone mode program \overlap is provided 
\begin{verbatim}
 moo_overlap --conjseg benzene.xml --pos1 benzene1.pos --pos2 benzene2.pos
\end{verbatim}
Its input requires a description of two conjugated segments (\texttt{benzene.xml}, positions and orientations of the molecules and the files with molecular coordinates and orbitals. The structure of the files is shown in listings \ref{list:benzene_xml} and  \ref{list:benzene_pos}.
\vskip 0.1cm
\lstinputlisting[label=list:benzene_xml,caption={\small \texttt{benzene.xml} file with the description of the benzene molecule, which is also a single conjugated segment and a rigid fragment.}] {./fig/moo/moo_overlap/benzene.xml}
\vskip 0.1cm
\lstinputlisting[label=list:benzene_pos, caption={\small \texttt{benzene1.pos} file which describes the position and orientation of the molecule. The name of the molecule is followed by three coordinates (relative to the center of mass of the supplied \texttt{xyz} file and then by nine elements of the rotation matrix $a_{ij} = e_i e^\text{mol}_j $. The reference coordinate frame is determined from the provided \texttt{xyz} file.}] {./fig/moo/moo_overlap/benzene1.pos}


\subsubsection{Calculator of \votcact}
Semi-empirical method of evaluation of electronic couplings is provided by the \integrals \calculator.


\moo requires the following input files: \\
\noindent
\xyz contains four columns, first being the atom type and the next three its coordinates. This is a standard \texttt{xyz} format without a header. 
\vskip 0.1cm
\noindent

\subsubsection{Molecular orbitals}
\orb can be generated using \gaussian program and the input script \texttt{get\_orbitals.com} which shown in listing~\ref{list:zindo_orbitals}.
\vskip 0.1cm
\noindent
\lstinputlisting[
 label=list:zindo_orbitals, 
 caption={\footnotesize \gaussian input file \texttt{get\_orbitals.com} used for generating molecular orbitals. The first line contains  the name of the check file, the second the requested RAM. 
%
 \texttt{int=zindos} requests the method ZINDO, \texttt{punch=mo} states that the molecular orbitals ought to be written to  the \texttt{fort.7} file, \texttt{nosymm} forbids use of symmetry and is necessary to ensure correct position of orbitals with respect to the provided coordinates. The two integer numbers correspond to the charge and multiplicity of the system: $0\, 1$ corresponds to a neutral system with a multiplicity of one. They are followed by the types and coordinates of all atoms in the molecule.
}]%
{./fig/moo/get_orbitals.com}
%
Provided with this input, \gaussian will generate \texttt{fort.7} file containing the molecular orbitals of a single molecule. This file can be renamed to \orb. 


\subsection{Density-functional method}
\label{sec:dft}

\begin{itemize}
\item {\it not sure about directory structure yet, using {\tt
      \$DIRECTORY} for the time being}
\item creating file structure for frame $N$ (raw: no postpocessing,
  min: MD energy minimization) in directory {\tt OUTDIR}
\begin{verbatim}
$DIRECTORY/perpare.sh raw/min N OUTDIR
cd OUTDIR
$DIRECTORY/pairdump.sh
\end{verbatim}
\item make sure QCP environments are set!
\item running calculations for all monomers
 \begin{verbatim}
$DIRECTORY/calc_monomer QCP [METHOD]

QCP:   G for Gaussian09
       T for Turbomole

METHOD: func/basis (optional)
        overrides default functional/basisset combination
        defaults: pbepbe/6-311G** Gaussian09
                  b-p/def-TZVP    Turbomole
\end{verbatim}
\item check monomer calculations (Gaussian version to test!) 
\begin{verbatim}
$DIRECTORY/check_mols N M QCP

N:   First monomer to test
M:   Last monomer to test
QCP: G/T 
\end{verbatim}
incomplete monomers are written to file {\tt TROUBLE.mol}
\item running calculations for all dimers
 \begin{verbatim}
$DIRECTORY/calc_dimer_noSCF QCP [METHOD]

QCP:   G for Gaussian09
       T for Turbomole

METHOD: func/basis (optional)
        overrides default functional/basisset combination
        defaults: pbepbe/6-311G** Gaussian09
                  b-p/def-TZVP    Turbomole
\end{verbatim}
\item should we add {\tt trajectory\_submit.sh} that does all monomer
  and dimer calculations on the cluster (MPIP-specific)?
\end{itemize}

