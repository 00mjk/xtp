\section{
Projection of monomer orbitals on dimer orbitals (DIPRO)
}\label{sec:idft}


\subsubsection{Notation}
$a$ - number (usual)\\
$\vctr{a}$ - vector  (bold,bar)\\
$\matr{A}$ - matrix (bold)\\
$\oper{A}$ - operator (usual, hat)

\subsubsection{Equations}

A nice Hamiltonian to keep in mind:

\begin{equation}
  \oper{H} = 
  \sum_m \epsilon_m \oper{a}_m^\dagger \oper{a}_m +
  \sum_{m \neq n} t_{nm} \oper{a}_m^\dagger \oper{a}_n + c.c.
  \label{eq:dipro_eq1}
\end{equation}

where $\oper{a}_m^\dagger$ and $\oper{a}_m$ are the creation and annihilation operators
for a charge carrier located at the molecular site $m$.
The electron site energy is given by $\epsilon_m$ , while $t_{nm}$ 
is the transfer integral between two sites $m$ and $n$. 

\textit{
The distribution in $\epsilon_m$
describes the density of states whereas $t_{nm}$ sets a timescale for
charge tunneling. (unnecessary)
}


We label the two individual molecules A
and B and their frontier orbitals $\phi_A$ and $\phi_B$, respectively.
We consider the  case
of hole transport, where the frontier orbital of interest is
the HOMO. (All arguments can be applied to electron transfer
by changing references to HOMOs to LUMOs.) 

We assume that the frontier orbitals of a dimer (adiabatic
energy surfaces) result exclusively from the interaction of the
frontier orbitals of monomers, and
can be expanded in terms of $\phi_A$ and $\phi_B$. The expansion
coefficients, $\vctr{C}$, can be determined by solving the
secular equation

\begin{equation}
  (\matr{H} - E \matr{S})\vctr{C} = 0
  \label{eq:dipro_eq2}
\end{equation}

where $\matr{H}$ and $\matr{S}$ are the Hamiltonian and overlap
matrices of the system, respectively. 

Since it is easier to work in matrix form, the following
equation also holds (equation (\ref{eq:dipro_eq2}) in matrix form):

\begin{equation}
 \matr{H}\matr{U} = \matr{S}\matr{U}\matr{E}
  \label{eq:dipro_eq3}
\end{equation}

These matrices can be written explicitly as

\begin{equation}
 \begin{aligned}
  \matr{H} = 
  \begin{pmatrix}
    e_A    &  J_{AB} \\
    J_{AB} &  e_B  
  \end{pmatrix} \\
  \matr{S} = 
  \begin{pmatrix}
    1    &  S_{AB} \\
    S_{AB} &  1  
  \end{pmatrix}
  \end{aligned}
  \label{eq:dipro_eq4}
\end{equation}

with (here we introduce even shorter notation)

\begin{equation}
 \begin{aligned}
  e_A = \Bra{\phi_A}\oper{H} \Ket{\phi_A} &= A\\
  e_B = \Bra{\phi_B}\oper{H} \Ket{\phi_B} &= B\\
  J_{AB} = \Bra{\phi_A}\oper{H} \Ket{\phi_B} &= J \\
  S_{AB} &= S 
 \end{aligned}
  \label{eq:dipro_eq5}
\end{equation}

In this new (shorter) notation :

\begin{equation}
 \begin{aligned}
  \matr{H} = 
  \begin{pmatrix}
    A    &  J \\
    J    &  B  
  \end{pmatrix} \\
  \matr{S} = 
  \begin{pmatrix}
    1    &  S \\
    S    &  1  
  \end{pmatrix}
  \end{aligned}
  \label{eq:dipro_eq6}
\end{equation}

\textbf{Note:} Even though it looks like matrices (and orbitals) are real,
it is just a bad language (for simplicity). 
All equations hold for complex wave functions, 
resulting in Hermitian matrices. 
Corresponding changes are obvious.

Here two
monomer HOMOs that form the basis of this expansion
are not orthogonal in general (but normalized), 
so that $\matr{S} \neq \matr{1}$.
It is therefore necessary to
transform equation (~\ref{eq:dipro_eq1}) into a standard eigenvalue problem
of the form

\begin{equation}
  \matr{H}^{\mathrm{eff}} \vctr{C}^{\mathrm{eff}} =
  E \vctr{C}^{\mathrm{eff}}
  \label{eq:dipro_eq7}
\end{equation}

or in matrix form

\begin{equation}
  \matr{H}^\mathrm{eff} \matr{U}^\mathrm{eff} =
  \matr{U}^\mathrm{eff} \matr{E}
  \label{eq:dipro_eq8}
\end{equation}

here $\matr{U}^\mathrm{eff}$ made of vectors $\vctr{C}^{\mathrm{eff}}$ 
and $\matr{E}$ is a diagonal matrix which contains eigenvalues (energies).
According to L\"owdin such a transformation can be
achieved by

\begin{equation}
  \matr{H^\mathrm{eff}} = \matr{S}^{\left. {-1} \middle/ {2} \right.}
  \matr{H}\matr{S}^{\left. {-1} \middle/ {2} \right.}
  \label{eq:dipro_eq9}
\end{equation}

with $\matr{S}^{\left. {-1} \middle/ {2} \right.}$, 
which can be calculated in standard (and efficient) way as

\begin{equation}
  f(\matr{S}) = \matr{U} f(\matr{S}_D) \matr{U^\dagger}
  \label{eq:dipro_eq10}
\end{equation}

where $\matr{S}_D$ is diagonalized version of $\matr{S}$, $\matr{U}$
is the corresponding unitary transformation and \\
$f(x)=x^{-\frac{1}{2}}$.

Applying this procedure to the $2\times2$ case in
eqn (~\ref{eq:dipro_eq1}) then yields an effective Hamiltonian matrix in
an orthogonal basis, and its entries can directly be
identified with the site energies $\epsilon_m$ 
and transfer integrals $t_{mn}$ (again we introduce
shortened notation, the backward restoration is straightforward):

\begin{equation}
 \begin{aligned}
  \matr{H}^{\mathrm{eff}} &= 
    \begin{pmatrix}
      A^{\mathrm{eff}}    &  J^\mathrm{eff} \\
      J\mathrm{eff}       &  B^\mathrm{eff}  
    \end{pmatrix} =
    \begin{pmatrix}
      \epsilon_a    &  t_{ab} \\
      t_{ab}       &  \epsilon_b  
    \end{pmatrix} =
    \begin{pmatrix}
      a    &  t \\
      t    &  b  
    \end{pmatrix}
 \end{aligned}
  \label{eq:dipro_eq11}
\end{equation}

The corresponding matrix elements are (in terms of A,B,S,J)

\begin{equation}
 \begin{aligned}
  a(b) &= \frac{1}{2}
    \frac{1}{1 - S^2} 
    ( (A + B) - 2 J S + 
    (A - B) \sqrt{1 - S^2} ) \\
  t &= \frac{J - \frac{1}{2}(A+B)S}{1-S^2}
 \end{aligned}
 \label{eq:dipro_eq12}
\end{equation}
 
energy $b$ can be obtained by interchanging $a$ and $b$ in eq 
(\ref{eq:dipro_eq12}).

It is apparent from eqn (\ref{eq:dipro_eq12}) that for orthogonal monomer
HOMOs, i.e. when $S$ = 0, the modifications to $A(B)$ and
$J$ vanish. In particular, this is the case for the commonly
used ZINDO approach. In the approximation that the orbitals
of the dimer $\psi^D$ are determined only by the HOMOs of
the monomers, the difference of the eigenvalues of the
effective Hamiltonian in eqn (\ref{eq:dipro_eq12}), 
$\Delta E$, is the value of the
splitting between the HOMO and HOMO-1 levels of
the dimer:

\begin{equation}
  \Delta E = 
  \sqrt{ (a - b)^2 + (2t)^2 }
   \label{eq:dipro_eq13}
\end{equation}

This indicates that the resulting energy split between HOMO
and HOMO-1 in the dimer can in general not be attributed
exclusively to the electronic coupling of the monomers
represented by $t$ but can also affected by the difference
in site energies, $a-b$. If one intends to determine the
transfer integral $t$ from the dimer energy splitting, one must
evaluate

\begin{equation}
  t = \frac{1}{2}
  \sqrt{ (\Delta E)^2 - (a - b)^2 }
   \label{eq:dipro_eq14}
\end{equation}

The difference of the site energies associated to the individual
monomers in the dimer configuration is only zero when 
the monomers are identical and their mutual geometric
orientation can be generated by symmetry transformations
that are compatible with the point group of the monomer,
i.e. when the dimer is symmetric. In a non-symmetric dimer,
monomer sites A and B polarize each other asymmetrically
and consequently contribute differently to the energetics of the
dimer. This also means that in general quantum-chemical
information is needed from both monomers and the dimer
to accurately determine the transfer integral.


Up to this point we have not mentioned in detail how to
suitably calculate the matrix elements $J_{AB}$. Direct numerical
integration in real space

\begin{equation}
  J = 
  \Bra{A} \oper{H} \Ket{B} = \int d \vctr{r} \:
  \phi^A (\vctr{r}) \oper{H} \phi^B (\vctr{r})
   \label{eq:dipro_eq15}
\end{equation}

is potentially inaccurate, this can be calculated (alternatively) 
via projections.

\begin{equation}
 \begin{aligned}
  \Ket{i} &= \Ket{\psi_i} \\
 \oper{1} &= \sum_i \Ket{i}\Bra{i} \\
 J &= 
 \Bra{A} \oper{H} \Ket{B} = 
 \Bra{A} \oper{1} \oper{H} \oper{1} \Ket{B} = 
 \sum_{ij} \Braket{A|i}\Bra{i}\oper{H}\Ket{j}\Braket{j|B} \\
 \Bra{i}\oper{H}\Ket{j} &= E_i \delta_{ij}
 \end{aligned}
  \label{eq:dipro_eq16}
\end{equation}

Introducing vectors of coefficients for states $\Ket{A}$ and $\Ket{B}$
as $\vctr{V}_A$ and $\vctr{V}_B$ respectively, we get

\begin{equation}
 \begin{aligned}
  V_{A}^{(i)} &= \Braket{A|i} \\
  V_{B}^{(i)} &= \Braket{B|i} \\
  J = 
  \vctr{V}_A^\dagger \matr{H}_D \vctr{V}_B &= 
  \vctr{V}_A^\dagger \matr{E}   \vctr{V}_B
 \end{aligned}
  \label{eq:dipro_eq17}
\end{equation}

we introduced here diagonalized Hamiltonian matrix 
$\matr{E} = \matr{H}_D$. 

What is left to do is determine these projections $\vctr{V}_{A(B)}$. 
Again it
is inconvenient to perform a real-space integration. In all
practical calculations the molecular orbitals are expanded in
basis sets of either plane waves or of localized atomic orbitals
$\Ket{i} = \Ket{\psi_i}$. We will first consider the case that 
the calculations for
the monomers are performed using a counterpoise (CP) basis
set that is commonly used to deal with the basis set super-
position error (BSSE) The basis set of atom-centered
orbitals of a monomer is extended to the one of the dimer
by adding the respective atomic orbitals at virtual coordinates
of the second monomer. We can then write the respective
expansions as

\begin{equation}
 \begin{aligned}
  \Ket{A} &= \sum_{\alpha} A_{\alpha} \Ket{\alpha} 
  \Longrightarrow 
  \vctr{A} = \{ A_{\alpha} \}
  \\
  \Ket{B} &= \sum_{\alpha} B_{\alpha} \Ket{\alpha}
  \Longrightarrow 
  \vctr{B} = \{ B_{\alpha} \}
  \\
  \Ket{i} &= \sum_{\alpha} D^{(i)}_{\alpha} \Ket{\alpha}
  \Longrightarrow 
  \vctr{D}^{(i)} = \{ D^{(i)}_{\alpha} \}
 \end{aligned}
  \label{eq:dipro_eq18}
\end{equation}

The projections can then be determined within the common
basis set:

\begin{equation}
 \begin{aligned}
    \Braket{A|i} = \sum_{\alpha} A_{\alpha} \Bra{\alpha}
    \sum_{\beta} D^{(i)}_{\beta} \Ket{\beta} = 
    \vctr{A}^\dagger \matr{S} \vctr{D}^{(i)} \\
    \Braket{B|i} = \sum_{\alpha} B_{\alpha} \Bra{\alpha}
    \sum_{\beta} D^{(i)}_{\beta} \Ket{\beta} = 
    \vctr{B}^\dagger \matr{S} \vctr{D}^{(i)} \\
 \end{aligned}
  \label{eq:dipro_eq19}
\end{equation}

leading to a final expression

\begin{equation}
 \begin{aligned}
    J = 
    \vctr{A}^\dagger \matr{S} \matr{D} \matr{E}
    \matr{D}^\dagger \matr{S}^\dagger \vctr{B}  \\
 \end{aligned}
  \label{eq:dipro_eq20}
\end{equation}


where the S is the overlap matrix of the atomic orbitals.
We summarize this methodology, which we will from now
on refer to as DIPRO, as the projection of monomer orbitals
on dimer orbitals, by the schematics in Fig. (\ref{fig:scheme}). 
After completion 
of the quantum-chemical calculations for both monomers
and dimers, the elements $J_{AB}$ can be calculated using eqn  
(\ref{eq:dipro_eq17})
by simple matrix-vector (or matrix-matrix) 
multiplications. From a quantum-chemical calculation the overlap
matrix of atomic orbitals $\matr{S}$, the expansion coefficients for
monomesrs $\vctr{A} = \{ A_{\alpha} \}$,$\vctr{B} = \{ B_{\alpha} \}$ 
 and dimer orbitals $\vctr{D}^{(i)} = \{ D^{(i)}_{\alpha} \}$, 
 as well as the orbital
energies $E_{i}$ of the dimer are required as input.

\begin{figure}[h]
    \center
    \includegraphics[width=0.8\textwidth]{fig/idft_flow/scheme_t}
    \caption{DIPRO scheme}
    \label{fig:scheme}
\end{figure}

Comparative study can be found here
~\cite{baumeier_density-functional_2010}.

