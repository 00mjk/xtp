\subsection{Projection of monomer orbitals on dimer orbitals (DIPRO)}
\label{sec:dipro}
An approach for the determination of the transfer integral that can be used for any single-particle electronic structure method (Hartree-Fock, DFT, or semiempirical methods) is based on the projection of monomer orbitals on a manifold of explicitly calculated dimer orbitals. This dimer projection (DIPRO) technique including an assessment of computational parameters such as the basis set, exchange-correlation functionals, and convergence criteria is presented in detail in ref.~\cite{baumeier_density-functional_2010}. A brief summary of the concept is given below.

We start from an effective Hamiltonian~\footnote{we use following notations: $a$ - number, $\vctr{a}$ - vector, $\matr{A}$ - matrix, $\oper{A}$ - operator}
%
\begin{equation}
  \oper{H}^\text{eff} = \sum_i \epsilon_i \oper{a}_i^\dagger \oper{a}_i + \sum_{j \neq i} J_{ij} \oper{a}_i^\dagger \oper{a}_j + c.c.
  \label{equ:dipro_eq1}
\end{equation}
%
where $\oper{a}_i^\dagger$ and $\oper{a}_i$ are the creation and annihilation operators for a charge carrier located at the molecular site $i$.
The electron site energy is given by $\epsilon_i$, while $J_{ij}$  is the transfer integral between two sites $i$ and $j$. We label their frontier orbitals (HOMO for hole transfer, LUMO for electron transfer) $\phi_i$ and $\phi_j$, respectively. Assuming that the frontier orbitals of a dimer (adiabatic energy surfaces) result exclusively from the interaction of the frontier orbitals of monomers, and consequently expand them in terms of $\phi_i$ and $\phi_j$. The expansion coefficients, $\vctr{C}$, can be determined by solving the secular equation
%
\begin{equation}
  (\matr{H} - E \matr{S})\vctr{C} = 0
  \label{equ:dipro_eq2}
\end{equation}
%
where $\matr{H}$ and $\matr{S}$ are the Hamiltonian and overlap matrices of the system, respectively. 
%
%Since it is easier to work in matrix form, the following
%equation also holds (equation (\ref{eq:dipro_eq2}) in matrix form):
%
%\begin{equation}
% \matr{H}\matr{U} = \matr{S}\matr{U}\matr{E}
%  \label{eq:dipro_eq3}
% \end{equation}
%
These matrices can be written explicitly as
%
\begin{equation}
% \begin{aligned}
  \matr{H} = 
  \begin{pmatrix}
    e_i    &  H_{ij} \\
    H_{ij}^* &  e_j  
  \end{pmatrix} \hspace{2cm}
  \matr{S} = 
  \begin{pmatrix}
    1    &  S_{ij} \\
    S_{ij}^* &  1  
  \end{pmatrix}
%  \end{aligned}
  \label{equ:dipro_eq3}
\end{equation}
%
with 
%
\begin{equation}
 \begin{aligned}
  e_i &= \Bra{\phi_i}\oper{H} \Ket{\phi_i} \hspace{2cm}  H_{ij} = \Bra{\phi_i}\oper{H} \Ket{\phi_j}\\
  e_j &= \Bra{\phi_j}\oper{H} \Ket{\phi_j} \hspace{2cm}  S_{ij} = \Bra{\phi_j} \phi_j\rangle %S 
 \end{aligned}
  \label{equ:dipro_eq4}
\end{equation}
The matrix elements $e_{i(j)}$, $H_{ij}$, and $S_{ij}$ entering \equ{dipro_eq3} can be calculated via projections on the dimer orbitals (eigenfunctions of $\hat{H}$) $\left\{\Ket{\phi^\text{D}_n}\right\}$ by inserting $\oper{1} = \sum_n \Ket{\phi^\text{D}_n}\Bra{\phi^\text{D}_n}$ twice. We exemplify this explicitly for $H_{ij}$ in the following
%
\begin{equation}
  H_{ij} = \sum_{nm}{\Braket{\phi_i|\phi^\text{D}_n} \Bra{\phi^{D}_n}\hat{H}\Ket{\phi^\text{D}_m}\Braket{\phi^\text{D}_m|\phi_j}} .
  \label{eq:dipro_eq16}
\end{equation}
%
The Hamiltonian is diagonal in its eigenfunctions, $\Bra{\phi^\text{D}_n}\oper{H}\Ket{\phi^\text{D}_m} = E_n \delta_{nm}$. Collecting the projections of the frontier orbitals  $\Ket{\phi_{i(j)}}$ on the $n$-th dimer state $\left(\vctr{V}_{(i)}\right)_n= \Braket{\phi_i|\phi^\text{D}_n}$ and $\left(\vctr{V}_{(j)}\right)_n=\Braket{\phi_j|\phi^\text{D}_n}$ respectively, into vectors we obtain

\begin{equation}
   H_{ij} = \vctr{V}_{(i)} \matr{E}   \vctr{V}_{(j)}^\dagger .
  \label{eq:dipro_eq17}
\end{equation}
%
What is left to do is determine these projections $\vctr{V}_{(k)}$. In all practical calculations the molecular orbitals are expanded in basis sets of either plane waves or of localized atomic orbitals $\Ket{\varphi_\alpha}$. We will first consider the case that the calculations for
the monomers are performed using a counterpoise basis set that is commonly used to deal with the basis set superposition error (BSSE). The basis set of atom-centered orbitals of a monomer is extended to the one of the dimer by adding the respective atomic orbitals at virtual coordinates of the second monomer. We can then write the respective expansions as

\begin{equation}
 %\begin{aligned}
  \Ket{\phi_{k}} = \sum_{\alpha} \lambda^{(k)}_\alpha \Ket{\varphi_\alpha} \hspace{1cm}\text{and}\hspace{1cm}
  \Ket{\phi^\text{D}_n} = \sum_{\alpha} D^{(n)}_\alpha \Ket{\varphi_\alpha}
  \label{eq:dipro_eq18}
\end{equation}
%
where $k=i,j$. The projections can then be determined within this common basis set as

 \begin{equation}
  \begin{aligned}
     \left(\vctr{V}_k\right)_n=\Braket{\phi_k|\phi^\text{D}_n} = \sum_{\alpha} \lambda^{(k)}_{\alpha} \Bra{\alpha} \sum_{\beta} D^{(n)}_{\beta} \Ket{\beta} = 
     \vctr{\boldsymbol{\lambda}}_{(k)}^\dagger \matr{\mathcal{S}} \vctr{D}_{(n)} 
%     %\\
% %    \Braket{B|i} = \sum_{\alpha} B_{\alpha} \Bra{\alpha}
% %    \sum_{\beta} D^{(i)}_{\beta} \Ket{\beta} = 
% %    \vctr{B}^\dagger \matr{S} \vctr{D}^{(i)} \\
  \end{aligned}
   \label{eq:dipro_eq19}
 \end{equation}
where $\matr{\mathcal{S}}$ is the overlap matrix of the atomic basis functions. This allows us to finally write the elements of the Hamiltonian and overlap matrices in \equ{dipro_eq3} as:

 \begin{equation}
  \begin{aligned}
     H_{ij} &= \vctr{\boldsymbol{\lambda}}_{(i)}^\dagger \matr{\mathcal{S}} \matr{D} \matr{E} \matr{D}^\dagger \matr{\mathcal{S}}^\dagger \vctr{\boldsymbol{\lambda}}_{(j)}  \\
     S_{ij} &= \vctr{\boldsymbol{\lambda}}_{(i)}^\dagger \matr{\mathcal{S}} \matr{D}  \matr{D}^\dagger \matr{\mathcal{S}}^\dagger \vctr{\boldsymbol{\lambda}}_{(j)} 
  \end{aligned}
   \label{eq:dipro_eq20}
 \end{equation}
%
Since the two monomer frontier orbitals that form the basis of this expansion are not orthogonal in general ($\matr{S} \neq \matr{1}$), it is necessary to transform \equ{dipro_eq2} into a standard eigenvalue problem of the form
%
\begin{equation}
  \matr{H}^{\mathrm{eff}} \vctr{C}^{\mathrm{eff}} =   E \vctr{C}^{\mathrm{eff}} 
  \label{eq:dipro_eq7}
\end{equation}
%
to make it correspond to \equ{dipro_eq1}. According to L\"owdin such a transformation can be achieved by
%
\begin{equation}
  \matr{H^\mathrm{eff}} = \matr{S}^{\left. {-1} \middle/ {2} \right.}
  \matr{H}\matr{S}^{\left. {-1} \middle/ {2} \right.}.
  \label{eq:dipro_eq9}
\end{equation}
%
This then yields an effective Hamiltonian matrix in an orthogonal basis, and its entries can directly be identified with the site energies $\epsilon_i$ and transfer integrals $J_{ij}$:
%
\begin{equation}
 \begin{aligned}
  \matr{H}^{\mathrm{eff}} &= 
    \begin{pmatrix}
      e_i^{\mathrm{eff}}    &  H_{ij}^\mathrm{eff} \\
      H_{ij}^{*,\mathrm{eff}}   &  e_j^\mathrm{eff}  
    \end{pmatrix} =
    \begin{pmatrix}
      \epsilon_i    &  J_{ij} \\
      J_{ij}^*      &  \epsilon_j  
    \end{pmatrix} 
 \end{aligned}
  \label{eq:dipro_eq11}
\end{equation}

 \begin{figure}[htb]
     \center
     \includegraphics[width=\linewidth]{fig/idft_flow/schemes_all}
     \caption{Schematics of the DIPRO method. (a) General workflow of the projection technique. (b) Strategy of the efficient noCP+noSCF implementation, in which the monomer calculations are performed independently form the dimer configurations (noCP), using the \calc{edft} \calculator. The dimer Hamiltonian is subsequently constructed based on an initial guess formed from monomer orbitals and only diagonalized once (noSCF) before the transfer integral is calculated by projection. This second step is performed by the \calc{idft} \calculator. }
     \label{fig:dipro_scheme}
 \end{figure}

