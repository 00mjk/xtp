\section{Random Facts}

This section is a collection of facts we discovered about xtp and ctp which should be included in the manual at some point, but lack proper background.



\subsection{xqmm}

The cutoffs used should not exceed the dimension of the cell, at least for a non orthogonal unit cell. XQMM throws an error if your cutoff is larger than the box, but it does not take the extension of the molecule into accout, so often you may still have overlap.


\subsection{EWALD}


\begin{itemize}
\item Use pewald3d, as a calculator. I am not sure what the rest is for. All still use the ewald options. The shape factor massively influences the results. For bulk systems "none" is the option of choice. Other options are "xyslab", "sphere", "cube" but I do not know what they do.
\item The induction cutoff should hardly ever exceed 3nm, because the calculation is expensive
\item If you want to use induction, you befor have to run ewdbgpol calculator and specify polar\textunderscore top.bgp in the options file for ewald. All the other parameters should be the same in ewdbpol and pewald3d
\end{itemize}
