\section{Semi-empirical determination of transfer intergals}
\label{sec:moo}

Molecular orbital overal module (MOO) is useful for fast evaluation of electronic coupling elements. It is based on the semi-empirical ZINDO Hamiltonian and therefore has limited applicability. The general advice is to first compare the accuracy of this method to the DFT-based calculations.  

Even though MOO is self-sufficent, it relies on the knowledge of the molecular orbitals of 

{\color{red} Old text from Thorsten. Cleaning is needeed.}

Two types of input files are necessary for each charge unit: a geometry specification (which we refer to as INPUT\_COORDS) and a molecular orbital file (which we refer to as fort.7).

First, let us have a quick look at the quantum mechanical input files: {\bf INPUT\_COORDS} and {\bf fort.7}. By convention, they should be stored in a folder named {\bf QM\_input}. INPUT\_COORDS is simply an \emph{xyz} file without headers, i.e. contains four columns, first being the atom type and the next three its coordinates. The {\bf fort.7} file is obtained by running a \emph{Gaussian} simulation with the input file {\bf get\_orbitals.com} as specified below.

\begin{verbatim}
%chk=zindo.chk
%mem=100Mb
#p int=zindos punch=mo nosymm

Norma CH3 calculation of orbitals

0 1
 C     0.094716     0.101569     0.002762
[...]
\end{verbatim}

The first line contains only the name of the check file,
the second the requested RAM. The third is the truly important
one explained below. After that follows a description of what
is being calculated. The two integer numbers correspond to the
charge and multiplicity of the system, $0 1$ therefore represents
a neutral system with a multiplicity of one. Below are the coordinates
and types of all atoms within the molecule given in \emph{xyz} format.

\begin{itemize}
 \item {\bf int=zindos} \\
        Requires the method \emph{ZINDO} to be used as integrator for the orbitals.
 \item {\bf punch=mo} \\
        States that the molecular orbitals ought to be written. Creates the {\bf fort.7} file.
 \item {\bf nosymm} \\
        Forbids use of symmetry simplifcations. Necessary to ensure correct order of orbitals.
\end{itemize}

Output generated are the {\bf fort.7} file containing the molecular orbitals as well as the file {\bf get\_orbitals.log} holding the usual Gaussian output. 

\subsubsection{The charge unit types: list\_charges.xml}


{\color{red} Old text from Thorsten. Cleaning is needeed.}

\begin{verbatim}
<crgunit_type>
        <ChargeUnitType>
                <posname>QM_files/INPUT_COORDS</posname>
                <orbname>QM_files/fort.7</orbname>
                <basis>INDO</basis>
                <transorb>64</transorb>
                <reorg>0.21</reorg>
                <name>NMC</name>
                <energy>-0.21705</energy>
                <monomer_atom_map>
                 21 23 13 0 1 2 3 4 5 6 7 8 9 10 11 12 14
                 15 16 17 18 19 20 22  24 25 26 27 28 30
                 31 32 33 34 35 36 37 38 39 40 41 42 43 44 45
                </monomer_atom_map>
                <monomer_atom_weights>
                  16 16 12 12 12 12 12 12 12 14 12 12 12 12 12 12 12 14
                  12 12 12 12 12 12 12 12 1 1 1 1 1
                  1 1 1 1 1 1 1 1 1 1 1 1 1 1 1
                </monomer_atom_weights>

        </ChargeUnitType>
</crgunit_type>
\end{verbatim}

\begin{itemize}
 \item {\bf posname} \\
 Location of {\bf INPUT\_COORDS}.
 \item {\bf orbname} \\
 Location of {\bf fort.7}.
 \item{\bf basis} \\
 This should be set to INDO, unless the fort.7 has been created using another basis set. In that case it must be set to an xml file setting the characteristics of the basis set.
 \item {\bf transorb} \\
 Number of HOMO (LUMO) orbital. Corresponds to the number of $\alpha$ electrons in the \emph{Gaussian} log-file {\bf get\_orbitals.log} minus one (since counting in C++ starts at zero) for the HOMO and the number of $\alpha$ electrons for the LUMO.
 \item {\bf reorg} \\
 Reorganization energy of the cation or anion in eV calculated via \emph{Gaussian}. mp2 should be used at least for anions.
 \item {\bf name} \\
 Name of the mapping of the molecule. Must correspond to CG mapping.
 \item {\bf energy} \\
 Energy of the HOMO/LUMO level
 \item {\bf monomer\_atom\_map} \\
 List of atom indices as they were specified in the \emph{Gaussian} input used to create the {\bf fort.7} file. \\
 Note: The first three values are important, since they must correspond to the first three atoms defined in the coarse-grained mapping, which are used to calculate two vectors indicating the orientation of the molecule. The third required vector is the eigenvector of the smallest eigenvalue of the gyration tensor, i.e. perpendicular to the planar core. \\
 Note: The number of molecules here may differ from that in the coarse-grained mapping, since for example only the core is important for transport and not the side chains, but it has to be the same number of atoms as in the \emph{Gaussian} input file otherwise overlap integral values will be terribly wrong.
\end{itemize}
