\section{Mapping file}
\label{sec:xmlmap}
The mapping file (referred here as \xmlcsg) is used by the program \ctpmap to convert an atomistic trajectory to a coarse-grained one. The coarse-grained trajectory contains positions, names, types, and orientations of rigid fragments. \xmlcsg contains definitions of rigid fragments (coarse-grained beads) and identifies to what conjugated segment a particular rigid fragment belongs. The description of the mapping options is given in table \ref{tab:map}. An example of \xmlcsg for a \dcvt molecule is shown in listing~\ref{list:map}. 
%
\begin{table}[h]
\label{tab:map}
\caption{Description of the \xml mapping file (\xmlcsg).}
\rowcolors{1}{invisiblegray}{white} {\footnotesize \input{reference/xml/map.xml}}
\end{table}

% Define new language for listings.
\lstdefinelanguage{MXML} {
   basicstyle=\ttfamily\scriptsize,
   sensitive=true,
   morecomment=[s][\color{gray}\rmfamily\itshape]{<!--}{-->}, 
   showstringspaces=false,
   numberstyle=\scriptsize,
   numberblanklines=true,
   showspaces=false,
   breaklines=true,
   showtabs=false,
   alsoletter={:},
   keywords = [1]
   { name,cg_molecule,cg_beads,cg_bead,crgunitname,bead,beads,type,topology,name,ident,maps,map,mapping,weights,position,qm,symmetry },
   keywordstyle={[1]\color{blue}},
}

\lstinputlisting[
 language=MXML,
 label=list:map,
 caption={Examle of \xmlcsg for \dcvt. Each rigid fragment (coarse-grained bead) is defined by a list of atoms. Atom numbers, names, and residue names should correspond to those used in \gromacs topology (see the corresponing listing \ref{list:pdb} of the pdb file).}]%
{./input/dcv2t/map.xml}
