\section{Molecular orbitals}
If semi-empirical method is used to calculate electronic coupling elements, molecular orbitals of all molecules shall be supplied. They can be generated using \gaussian program. The \gaussian input file for \dcvt is shown in listing~\ref{list:zindo_orbitals}. Provided with this input, \gaussian will generate \texttt{fort.7} file which contains the molecular orbitals of a \dcvt. This file can be renamed to \texttt{\dcvt.orb}. Note that the order of the atoms in the input file and the order of coefficients should always match. Therefore, the coordinate part of the input file must be supplied together with the orbitals. We will assume the coordinates, in the format \texttt{atom\_type: x y z}, is saved to the \texttt{\dcvt.xyz} file.

\clearpage
\vskip 0.1cm
\noindent
\lstinputlisting[
 label=list:zindo_orbitals, 
 basicstyle=\ttfamily\small,
 morekeywords={chk,mem,punch,int,S,C,S,N,H},
 showstringspaces=false,
 keepspaces=true,
 caption={\small \gaussian input file \texttt{get\_orbitals.com} used for generating molecular orbitals. The first line contains  the name of the check file, the second the requested RAM. 
%
 \texttt{int=zindos} requests the method ZINDO, \texttt{punch=mo} states that the molecular orbitals ought to be written to  the \texttt{fort.7} file, \texttt{nosymm} forbids use of symmetry and is necessary to ensure correct position of orbitals with respect to the provided coordinates. The two integer numbers correspond to the charge and multiplicity of the system: $0\, 1$ corresponds to a neutral system with a multiplicity of one. They are followed by the types and coordinates of all atoms in the molecule.
}]%
{./input/get_orbitals.com}
%

